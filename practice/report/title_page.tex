\begin{titlepage}
    \centering

    \vspace*{3cm}

    % --- Title (16pt, bold) ---
    {\fontsize{16pt}{18pt}\selectfont \textbf{Enterprise Internship (SVNC.00.262)}\par}

    \vspace{1cm}

    % --- Subtitle (16pt, NOT bold unless required) ---
    {\fontsize{16pt}{18pt}\selectfont 
    Unpaid Internship In Students of Satellite  07.01.2025\par}

    \vspace{3cm}

    % --- Author / Institution (12pt default) ---
    {\fontsize{12pt}{14pt}\selectfont
    \textbf{Sergei Ivanov}\par
    \vspace{0.3cm}
    \textit{Information Technology Systems Development, University of Tartu}\par
    }

    \vspace{2cm}

    % --- Instructor and date ---
    {\fontsize{12pt}{14pt}\selectfont
    \textbf{Supervisor: Andre Sääsk}\par
    \vspace{0.3cm}
    \today\par
    }

    \vfill

\end{titlepage}


\begin{abstract}
This report describes the design and implementation of a Discord-integrated project management system developed during an internship at a software company. The system was created to improve internal coordination by unifying organizational workflows, task notifications, and data persistence within a familiar communication platform. Implemented primarily in Rust, the solution consists of a Discord bot interface, a Jira-based reminder service, and a PostgreSQL-backed persistence layer, with optional data synchronization to Google Sheets for non-technical stakeholders. The architecture follows containerized, microservice-ready principles and emphasizes memory safety, concurrency, and secure API usage. Throughout the internship, the system was developed iteratively under weekly supervision, addressing real-world constraints such as restricted API access and deployment consistency. The final result is a scalable, production-ready automation tool that reduces email dependency and enhances team visibility into events, resources, and task status.
\end{abstract}

\vspace{0.5cm}

\noindent\textbf{Keywords:} Discord bot, Rust programming language, project management automation, Jira integration, PostgreSQL, microservices, Docker, REST API



\tableofcontents
\clearpage


