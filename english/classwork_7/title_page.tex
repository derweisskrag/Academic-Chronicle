\begin{titlepage}
    \centering % Center everything on the page

    % --- Title Section ---
    \vspace*{3cm} % Push content down
    \Huge\textbf{ENGLISH FOR SPECIFIC PURPOSES (IT)}\par % Main Title
    \vspace{1cm}
    \Large Classwork Unit 7: Technology and Current Events\par % Subtitle

    \vspace{3cm} % Space between title and author info

    % --- Author/Affiliation Section ---
    \large\textbf{Sergei Ivanov}\par % Author Name
    \vspace{0.3cm}
    \textit{IT Systems development, University of Tartu}\par % Affiliation
    
    \vspace{2cm} % Space between author and instructor info
    
    % --- Instructor/Date Section ---
    \large\par
    \textbf{Instructor: Kayleigh Kleiva}\par % Instructor Name
    \vspace{0.3cm}
    \today\par % Date
    
    % You can also use the variable from YAML if you prefer the static date:
    % \large 28.10.2025\par 
    
    % --- Bottom of Page Space ---
    \vfill % Push remaining content to the bottom (optional)

\end{titlepage}


\begin{abstract}
The article analyzes the recent announcement of Rust version 1.91.0 and 
highlights its major contributions to the stability, performance, and usability
 of the language. It outlines the stabilization of key language features, 
 enhancements to the standard library, and improvements to the compiler and 
 tooling ecosystem. The summary also explains how these updates strengthen Rust's
  position as a safe and efficient programming language suitable for 
  both industry and open-source development. Through examining the main ideas, 
  supporting details, and newly encountered terminology, 
  the summary demonstrates how the release contributes to 
  the language's long-term growth and technical maturity.
\end{abstract}

\vspace{0.5cm}

\noindent\textbf{Keywords:}Rust programming language; compiler optimizations; stabilization; nightly channel; zero-cost abstractions; language features; software development tools

\tableofcontents
\clearpage