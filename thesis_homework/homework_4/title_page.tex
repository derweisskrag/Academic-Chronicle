\begin{titlepage}
    \centering

    \vspace*{3cm}

    % --- Title (16pt, bold) ---
    {\fontsize{16pt}{18pt}\selectfont \textbf{Teadustöö alused (P2NC.01.018)}\par}

    \vspace{1cm}

    % --- Subtitle (16pt, NOT bold unless required) ---
    {\fontsize{16pt}{18pt}\selectfont 
    Ülesanne 6 Refereerimine 20.11.2025\par}

    \vspace{3cm}

    % --- Author / Institution (12pt default) ---
    {\fontsize{12pt}{14pt}\selectfont
    \textbf{Sergei Ivanov}\par
    \vspace{0.3cm}
    \textit{IT Süsteemide arendus, Tartu Ülikool}\par
    }

    \vspace{2cm}

    % --- Instructor and date ---
    {\fontsize{12pt}{14pt}\selectfont
    \textbf{Õppejõud: Jelena Rootamm-Valter}\par
    \vspace{0.3cm}
    \today\par
    }

    \vfill

\end{titlepage}

\renewcommand{\abstractname}{Kokkuvõte}
\begin{abstract}
See töö annab ülevaate valitud teadusliku teabe kasutamisest ja struktureerimisest akadeemilises kirjalikus töös.
 Kokkuvõttes selgitatakse teadusfakti mõistet, milleks peetakse süstematiseeritud ja töödeldud üksikfaktide kogumit,
  ning tuuakse näide greedy algoritmi teaduslikust määratlusest, rõhutades selle rolli optimeerimisprobleemide lahendamisel. 
  Lisaks kirjeldatakse lõputööga seotud võtmeteemasid, nagu WebAssembly, Cython, andmestruktuurid, algoritmid, rakendused ja 
  jõudlusnäitajad, mis moodustavad uurimuse teoreetilise ja praktilise raamistiku. Esitatud teabeallikad toetavad väiteid ja 
  pakuvad aluse edasiseks analüüsiks ning meetodite rakendamiseks.
 Kokkuvõte näitab, et kui teaduslik info on korrektselt viidatud ja süsteemselt esitatud, siis on võimalik luua terviklik ja usaldusväärne akadeemiline arutlus.
\end{abstract}

\vspace{0.5cm}

\noindent\textbf{Võtmesõnad:}WebAssembly; Cython; andmestruktuurid; algoritmid; greedy-algoritm; jõudlusnäitajad; rakendused; määratlus;

\renewcommand{\contentsname}{Sisukord}
\tableofcontents
\clearpage
