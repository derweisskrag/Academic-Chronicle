\begin{titlepage}
    \centering

    \vspace*{3cm}

    % --- Title (16pt, bold) ---
    {\fontsize{16pt}{18pt}\selectfont \textbf{Lõputöö}\par}

    \vspace{1cm}

    % --- Subtitle (16pt, NOT bold unless required) ---
    {\fontsize{16pt}{18pt}\selectfont 
    Tõhusate Andmestruktuuride ja Algoritmide Arendamine Veebirakendustele: 
    Pythoni Jõudluse Optimeerimine WebAssembly (Wasm) abil\par}

    \vspace{3cm}

    % --- Author / Institution (12pt default) ---
    {\fontsize{12pt}{14pt}\selectfont
    \textbf{Sergei Ivanov}\par
    \vspace{0.3cm}
    \textit{IT Süsteemide arendus, Tartu Ülikool}\par
    }

    \vspace{2cm}

    % --- Instructor and date ---
    {\fontsize{12pt}{14pt}\selectfont
    \textbf{Juhendaja: Andre Sääsk}\par
    \vspace{0.3cm}
    \today\par
    }

    \vfill

\end{titlepage}

\renewcommand{\abstractname}{Kokkuvõte}
\begin{abstract}
Akadeemiline artikkel räägib mikroteenustest, täpsemalt sellest, kuidas nad lõid oma SpringBoot põhirakendusele Rust/Go lahenduse.
See väidab, et Python on aeglane programmeerimiskeel isegi siis, kui rakenduse jõudluse kiirendamiseks saab kasutada JIT-kompilaatorit. 
Sisuliselt ei toeta kõik Pythoni teegid JIT-i. Go-d kasutati WASM-i jaoks plugeeriva ekstismi loomiseks. Rust oli API loomise algoritmi peamine keel, kasutades `powerbot::api`. 
Lõpuks näitab artikkel mikroteenuste API loogika jaoks Go/Rust wasm-i kasutava rakenduse jõudluse kvantitatiivset analüüsi. See analüüs näitas, et Python on Rusti API-ga võrreldes väga aeglane. WASM-i eeliseks Kotlini SpringBoot rakenduse puhul on turvalisus ja kiirus, kuna artikli autor kasutas mikroteenuste vaheliste turvatud ühenduste jaoks TLS-sertifikaati.
\end{abstract}

\vspace{0.5cm}

\noindent\textbf{Võtmesõnad:} Wasm, Rust, Python, GO, API, WASI, monolith, microservices, lower level language, optimization, performance, scalability, agility, container, infrastructure, function, command, STOMP Protocol

\renewcommand{\contentsname}{Sisukord}
\tableofcontents
\clearpage
