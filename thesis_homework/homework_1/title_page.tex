\begin{titlepage}
    \centering % Center everything on the page

    % --- Title Section ---
    \vspace*{3cm} % Push content down
    \Huge\textbf{Lõputöö}\par % Main Title
    \vspace{1cm}
    \Large Tõhusate Andmestruktuuride ja Algoritmide Arendamine Veebirakendustele: Pythoni Jõudluse Optimeerimine WebAssembly (Wasm) abil\par % Subtitle

    \vspace{3cm} % Space between title and author info

    % --- Author/Affiliation Section ---
    \large\textbf{Sergei Ivanov}\par % Author Name
    \vspace{0.3cm}
    \textit{IT Süsteemide arendus, Tartu Ülikooli}\par % Affiliation
    
    \vspace{2cm} % Space between author and instructor info
    
    % --- Instructor/Date Section ---
    \large\par
    \textbf{Juhendaja: Andre Sääsk}\par % Instructor Name
    \vspace{0.3cm}
    \today\par % Date
    
    % You can also use the variable from YAML if you prefer the static date:
    % \large 28.10.2025\par 
    
    % --- Bottom of Page Space ---
    \vfill % Push remaining content to the bottom (optional)

\end{titlepage}

\renewcommand{\abstractname}{Kokkuvõte}
\begin{abstract}
Käesolev lõputöö uurib tõhusate andmestruktuuride ja algoritmide arendamist veebirakenduste jaoks, keskendudes Pythoni jõudluse optimeerimisele WebAssembly (Wasm) abil. Uurimistöö hindab Wasm-moodulite integreerimist Pythoni-põhistesse süsteemidesse ning mõõdab jõudluse paranemist reaalse kasutuse kontekstis.
\end{abstract}

\vspace{0.5cm}

\noindent\textbf{Võtmesõnad:} WebAssembly, Cythoni integreerimisega, andmestruktuur, algoritmid, rakendus, jõudlusnäitajaid

\renewcommand{\contentsname}{Sisukord}
\tableofcontents
\clearpage